\chapter{جمع‌بندی}
 به طور خلاصه در این پژوهش ما پیشرفت‌های اخیر خلاصه‌سازی متن انتزاعی را با استفاده از مدل‌های یادگیری عمیق، مدل‌های یادگیری تقویتی و مدل‌های مبتنی بر ساختار بررسی کرده‌ایم. ما در مورد رویکرد های مختلفی که پیشنهاد شده است و چالش هایی که هنوز باید پرداخته شوند، بحث کرده‌ایم.

مدل‌های یادگیری عمیق به ویژه هنگامی که با یادگیری تقویتی ترکیب شوند، برای خلاصه‌سازی متن انتزاعی  مؤثر هستند. با این که، آموزش این مدل‌ها  از نظر محاسباتی پرهزینه است و خلاصه‌ی تولید شده توسط این مدل‌های ممکن است  به اندازه خلاصه‌های نوشته شده توسط انسان روان یا آموزنده نباشد زیرا مدل‌های یادگیری عمیق بر روی مجموعه داده‌های بزرگی دادگان آموزش داده می‌شوند که جمع‌آوری و برچسب‌گذاری آن زمان‌بر و پرهزینه است.

مدل‌های مبتنی بر ساختار پتانسیل رفع برخی از محدودیت‌های مدل‌های یادگیری عمیق را دارند. این مدل‌ها می‌توانند دانش حوزه و قواعد زبانی را در خود جای دهند که می‌تواند به بهبود کیفیت خلاصه‌های تولید شده کمک کند. با این حال، آموزش این مدل ها دشوار است و قابل تعمیم به حوزه‌های جدید نیست.

به طور کلی، در سال های اخیر پیشرفت قابل توجهی در خلاصه سازی متن انتزاعی حاصل شده است و رویکردهای مختلفی برای کاهش افزونگی و افزایش خوانایی خلاصه‌ها ارائه شده است با این حال، هنوز چالش‌هایی زیادی وجود دارد.در حال حاضر، تحقیقات خلاصه سازی انتزاعی  بر یافتن مناسب‌ترین مدل‌های از پیش آموزش دیده و چگونگی تطبیق بازنمایی‌های به دست آمده از این مدل‌ها برای بهبود  کیفیت خلاصه‌ها و نزدیک‌تر کردن آن‌ها به  خلاصه‌سازی انسانی متمرکز است.
