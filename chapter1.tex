\chapter{مقدمه}
\section{مقدمه}
با رشد روزافزون اینترنت، حجم محتوای متنی در اینترنت (به عنوان مثال وب سایت‌ها، اخبار، وبلاگ‌ها، شبکه‌های رسانه‌های اجتماعی و غیره )به صورت تصاعدی افزایش می‌یابد. در نتیجه، کاربران زمان زیادی را صرف یافتن اطلاعات مورد نظر خود می‌کنند و نمی‌توانند تمام محتوای متنی نتایج جستجو را بخوانند. خلاصه‌سازی خودکار اسناد می‌تواند به شناسایی مهم‌ترین اطلاعات و صرفه‌جویی در وقت خوانندگان کمک کند. خلاصه‌سازی خودکار متن فرآیند تولید یک متن کوتاه است که بخش‌های اصلی یک سند طولانی‌تر را پوشش می‌دهد. یک خلاصه خوب جنبه‌های مهمی مانند خوانایی، انسجام، نحو، غیر زائد بودن، ترتیب جملات، مختصر بودن، تنوع اطلاعات و پوشش اطلاعات را در نظر می‌گیرد
\cite{ELKASSAS2021113679}.


در سال‌های گذشته تلاش‌های زیادی برای تولید خلاصه‌سازی خودکار قابل قبول و خوانا صورت گرفته است. 
پژوهش‌های مرتبط با عمل خلاصه‌سازی خودکار متن در دهه ۵۰ میلادی شکل گرفتند. در یکی از این پژوهش‌ها لوهن و همکاران روشی برای خلاصه‌سازی اسناد علمی ‌ارائه دادند که در آن تابعی بر اساس فرکانس تکرار کلمات یا عبارات به عنوان ویژگی تعریف ‌می‌شود و وزن‌های مرتبط با این ویژگی ها خلاصه استخراج می‌شود
\cite{luhn1958automatic}.
در کارهای تحقیقاتی اولیه، مدل‌های غیرعصبی  مبتنی بر ساختار برای تولید خلاصه‌سازی خودکار مورد استفاده قرار گرفتند. با شروع دوره‌ی شبکه‌های عصبی عمیق پژوهش‌ها بر روی خلاصه‌سازی بیشتر شد. رویکرد‌های نوین خلاصه‌سازی شامل شبکه‌های عصبی عمیق دنباله به دنباله\LTRfootnote{Deep neural sequence to sequence models}
، روش‌های بر پایه‌ی مدل ترنسفورمر\LTRfootnote{transformer} 
و مدل‌های زبانی از پیش آموزش دیده\LTRfootnote{Pretrained language models (PTLMs)}
می‌باشد. 
%همچنین برخی از پژوهش های اخیر نشان داده‌اند، استفاده از رویکردهای مبتنی بر یادگیری تقویتی\LTRfootnote{ reinforcement learning (RL)}
%می‌تواند موجب بهبود معیارهای مختلف، از جمله امتیازات روژ، کیفیت کلی، خوانایی، انسجام، نحو، غیر افزونگی، ترتیب جملات، مختصر بودن، تنوع اطلاعات، پوشش اطلاعات شود.
\section{چالش‌های خلاصه‌سازی‌متون طولانی}
 
\section{چشم‌انداز نوشتار}

%در این گزارش مروری بر مدل‌های ارائه شده برای خلاصه‌سازی انتزاعی ارائه می‌کنیم. در بخش دوم چندین روش مبتنی بر ساختار برای خلاصه سازی انتزاعی مورد بررسی قرار می‌گیرد. در فصل سوم روش‌های مبتنی بر شبکه‌های عصبی برای خلاصه سازی انتزاعی از جمله روش‌های مبتنی بر مدل‌های کدگذار-کدگشا و ترنسفورمر بررسی می‌شوند. در فصل چهارم روش‌های مبتنی بر یادگیری تقویتی مورد بررسی قرار می‌گیرد و در نهایت در فصل پنجم جمع‌بندی مطرح می‌شود.